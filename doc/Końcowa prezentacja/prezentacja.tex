\documentclass{beamer}
\usetheme{Boadilla}
\usepackage{polski}
\usepackage[utf8]{inputenc}
\title{BTS Maintenance System }
\subtitle{Podsumowanie}
 \author{Michał Błach, Kacper Szewczyk}
%\institute{Slyboots \& Zawodnik Company}
\date{\today}
\setbeamercovered{transparent}
\begin{document}
\begin{frame}
\titlepage
\end{frame}
\begin{frame}
\tableofcontents
\end{frame}
\section{Ambitne początki}

\subsection{Wstępne założenia}
\begin{frame}{Znana sytuacja ?}
     \begin{columns}[T] % contents are top vertically aligned
      \begin{column}[T]{8cm} % alternative top-align that's better for graphics
          \includegraphics[width=8cm]{manVsDragon.jpg}
     \end{column}
     \begin{column}[T]{4cm} % each column can also be its own environment
     \begin{block}<1->{Nowy projekt/zlecenie}
	Rozpoczynamy nowy projekt startup.\\
	Jesteśmy pełni optymizmu i gotowi do działania. Zabieramy się za robotę, ale\dots 
\end{block}
     \end{column}
     \end{columns}
\end{frame}

\subsection{Brutalna rzeczywistość}
\begin{frame}{Znana sytuacja ?}
     \begin{columns}[T] % contents are top vertically aligned
     
          \begin{column}[T]{7cm} % each column can also be its own environment
     \begin{block}<1->{Założenia, a rzeczywistość}
    %\framesubtitle{<subtitle>}
	\begin{itemize}
    \item<1-> HTML może i nie jest bardzo skomplikowany\dots
    \item<2-> Ale ustawienie obiektów przy pomocy CSS, żeby porządnie wyglądało na różnych urządzenia to już spory problem
    \item<3-> Tak samo z JavaScript...
    \item<4-> O znajdywaniu działających na wielu urządzeniach pluginów, które wciąż są rozwijane to już nie wspomnimy
    \item<5->\dots
    \end{itemize}
\end{block}
     \end{column}
      \begin{column}[T]{5cm} % alternative top-align that's better for graphics
          \includegraphics[width=5cm]{manVsDragon2.jpg}
     \end{column}
     \end{columns}
\end{frame}


\begin{frame}{Plan działania}
     \begin{columns}[T] % contents are top vertically aligned
     
          \begin{column}[T]{5cm} % each column can also be its own environment
     \begin{block}<1->{Założony czas działania}
    %\framesubtitle{<subtitle>}
	\begin{itemize}
    \item<1-> Nauka HTML5, CSS, JavaScript - 2 tygodnie
    \item<2-> Wybór technologii - 2 tygodnie
    \item<3-> Szablon GUI - 2 tygodnie
    \item<4-> Podpięcie JavaScript pod strony - 1 tydzień
    \item<5-> SMSowe Api - 1 tydzień
    \item<6-> Połaczenie z bazą danych - 2 tygodnie
    \end{itemize}
\end{block}
     \end{column}
    \begin{column}[T]{5cm}
       \begin{block}<7->{Rzeczywisty}
    %\framesubtitle{<subtitle>}
	\begin{itemize}
    \item<8-> Nauka HTML5, CSS, JavaScript - \textbf{cały projekt}
    \item<9-> Wybór technologii - \textbf{4 tygodnie}
    \item<10-> Szablon GUI - \textbf{4 tygodnie}
    \item<11-> Podpięcie logiki stron w JavaScript - \textbf{2 dni}
    \item<12-> SMSowe Api -  \textbf{1 dzień}
    \item<13-> Połaczenie z bazą danych - \textbf{3.5 tygodnie}
    \end{itemize}
\end{block}
     \end{column}
     \end{columns}
\end{frame}
\section{Produkt}
\begin{frame}{Produkt trzeba dobrze sprzedać}
 \includegraphics[width=10cm]{paczka.jpg}
 \end{frame}
\section{Plany na przyszłość}
\begin{frame}{A co po egzaminie?}
\dots 
\end{frame}
\begin{frame}{Koniec}
 \includegraphics[width=10cm]{nicolas.jpg}
\end{frame}
\end{document}
