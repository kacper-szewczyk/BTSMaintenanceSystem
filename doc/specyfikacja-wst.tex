\documentclass[12pt]{article}
\usepackage{polski}
\usepackage[utf8]{inputenc}

\title{Wstępna specyfikacja projektu}
\date{12 października 2015}
\newcommand{\tab}[1]{\hspace{.05\textwidth}\rlap{#1}}
\begin{document}

\maketitle

\section{Skład grupy}
Grupa składa się z dwóch osób:
\begin{itemize}
\item Kacper Szewczyk(PM) 192789
\item Michał Błach 194127
\end{itemize}
\section{Temat projektu}
\textbf{"Stworzenie systemu dla pracowników z działu utrzymania łączności bezprzewodowej"}
\section{Wstępny zakres i cele projektu}
Celem projektu jest stworzenie systemu , który ułatwiałby prace ludziom z działu utrzymania łączności bezprzewodowej.

Aplikacja ta będzie zawierała bazę danych stacji bazowych telefonii komórkowych (BTS-ów),które należą do utrzymania danej firmy.

Użytkownik powinien w łatwy sposób znaleźć szukaną stację po którymś z jej parametrów, jak również mieć możliwość sprawdzenia listy najbliższych stacji korzystając z modułu GPS.

System powinien udostępniać w łatwy sposób dostęp do wszystkich danych stacji, jak również do szybkiej obsługi API operatora.

\section{Specyfikacja wymagań}
\textbf{Informacje, które będą mogły być przechowywane:}
\begin{enumerate}
\item Wewnętrzne id stacji u operatora/operatorów
\item Nazwa stacji u operatora/operatorów 
\item Właściciel stacji
\item Adres stacji
\item Typ stacji( np wieża , komin ...)
\item Koordynaty
\item Wysokość
\item Opis dostępu
\item Adres zakładu dostaw nośników energetycznych, który zasila stacje
\item Nr telefonu zakładu dostaw nośników energetycznych, pod który należy dany obiekt
\item Imię i nazwisko osoby , która była ostatnio na stacji
\item Data ostatniej aktywności na stacji
\end{enumerate}

Program będzie pozwalał wyszukiwać stacje, według dowolnego jej parametru jak również odczytywać obecne położenie telefonu i znajdować najbliższą możliwą.\\
\textbf{Po odnalezieniu szukanej stacji, użytkownik będzie mógł:}
\begin{enumerate}
\item Sprawdzić wszystkie informacje o niej
\item Wybrać opcje automatycznego włączenia nawigacji w telefonie, która będzie prowadziła go na adres obiektu.
\item Szybko wybrać numer zakładu dostaw nośników energetycznych, do którego należy dana stacja.
\end{enumerate}
\textbf{Również poprzez wysłanie smsa o odpowiedniej treści do operatora użytkownik będzie mógł:}
\begin{enumerate}
\item Zgłosić wejście na stacji (wystawić wejściówke)
\item Zgłosić wyjście ze stacji (zamknąć wejściówke)
\item Sprawdzić stan alarmów na stacji.
\end{enumerate}

\textbf{W skład projektu będzie wchodził:}
\begin{enumerate}
\item Aplikacja na urządzenia mobilne (platforma Android, WindowsPhone, iOS) napisana we frameworku PhoneGap/Mosync/Corodova.
\item Dedykowany serwer, który będzie udostępniał API do synchronizacji danych z urządzeniami mobilnymi(*)
\item Aplikacja webowa, która będzie pozwalała na szybkie dodawanie, edytowanie i usuwania rekordów z bazy danych. Będzie także dawała możliwość ustalenia dyżurów w pracy, jak również zlokalizowanie najbliższego wolnego pracownika od miejsca awarii stacji(*).
\end{enumerate}
(*) - opcjonalne
\section{WBS}
\section{Harmonogram projektu}
\begin{figure}[h]
    %\includegraphics[width=0.5\textwidth]{tomography-principle}
\end{figure}
\end{document}