\documentclass{beamer}
\usetheme{Boadilla}
\usepackage{polski}
\usepackage[utf8]{inputenc}
\title{BTS Maintenance System }
\subtitle{Wykorzystane technologie}
 \author{Michał Błach, Kacper Szewczyk}
%\institute{Slyboots \& Zawodnik Company}
\date{\today}

\begin{document}
\begin{frame}
\titlepage
\end{frame}
\begin{frame}
\tableofcontents
\end{frame}

\section{Aplikacja kliencka}
\begin{frame}
\frametitle{Frame1}
\end{frame}
\subsection{Problem z wyborem technologii}
\begin{frame}
\frametitle{Frame1}
\end{frame}
\subsection{Dostępne frameworki}
\begin{frame}
\frametitle{Frame1}
\end{frame}
\subsection{Cordova}
\begin{frame}
\frametitle{Frame1}
\end{frame}
\subsubsection{HTML5}
\begin{frame}
\frametitle{Frame1}
\end{frame}
\subsubsection{CSS3}
\begin{frame}
\frametitle{Frame1}
\end{frame}
\subsubsection{Java Script}
\begin{frame}
\frametitle{Frame1}
\end{frame}
\section{Zewnętrzny serwer do synchronizacji danych}
\begin{frame}
\frametitle{Frame2}
\end{frame}
\section{Webowy klient do zarzadania danymi na serwerze zewnętrznym}
\begin{frame}
\frametitle{Frame2}
\end{frame}
\begin{frame}
\frametitle{What Are Prime Numbers?}
\begin{definition}
A \alert{prime number} is a number that has exactly two divisors.
\end{definition}
\end{frame}

\begin{frame}
\frametitle{There Is No Largest Prime Number}
\framesubtitle{The proof uses \textit{reductio ad absurdum}.}
\begin{theorem}
There is no largest prime number.
\end{theorem}
\begin{proof}
\begin{enumerate}
\item<1-> Suppose $p$ were the largest prime number.
\item<2-> Let $q$ be the product of the first $p$ numbers.
\item<3-> Then $q + 1$ is not divisible by any of them.
\item<1-> But $q + 1$ is greater than $1$, thus divisible by some prime
number not in the first $p$ numbers.\qedhere
\end{enumerate}
\end{proof}
\uncover<4->{The proof used \textit{reductio ad absurdum}.}
\end{frame}

\end{document}



\end{document}
